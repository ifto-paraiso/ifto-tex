%%%%%%%%%%%%%%%%%%%%%%%%%%%%%%%%%%%%%%%%%%%%%%%%%%%%%%%%%%%%%%%%%%%%%%%%%%%%%%%%
% 
% IFTO-Tex (v1.0)
% https://github.com/renatouchoa/ifto-tex
%
% MODELO DO TRABALHO DE CONCLUSÃO DE CURSO DO IFTO
% 
%%%%%%%%%%%%%%%%%%%%%%%%%%%%%%%%%%%%%%%%%%%%%%%%%%%%%%%%%%%%%%%%%%%%%%%%%%%%%%%%
% 
% MIT License
% 
% Copyright (c) 2016 Renato Uchôa Brandão
% 
% Permission is hereby granted, free of charge, to any person obtaining a copy
% of this software and associated documentation files (the "Software"), to deal
% in the Software without restriction, including without limitation the rights
% to use, copy, modify, merge, publish, distribute, sublicense, and/or sell
% copies of the Software, and to permit persons to whom the Software is
% furnished to do so, subject to the following conditions:
% 
% The above copyright notice and this permission notice shall be included in all
% copies or substantial portions of the Software.
% 
% THE SOFTWARE IS PROVIDED "AS IS", WITHOUT WARRANTY OF ANY KIND, EXPRESS OR
% IMPLIED, INCLUDING BUT NOT LIMITED TO THE WARRANTIES OF MERCHANTABILITY,
% FITNESS FOR A PARTICULAR PURPOSE AND NONINFRINGEMENT. IN NO EVENT SHALL THE
% AUTHORS OR COPYRIGHT HOLDERS BE LIABLE FOR ANY CLAIM, DAMAGES OR OTHER
% LIABILITY, WHETHER IN AN ACTION OF CONTRACT, TORT OR OTHERWISE, ARISING FROM,
% OUT OF OR IN CONNECTION WITH THE SOFTWARE OR THE USE OR OTHER DEALINGS IN THE
% SOFTWARE.
%
%%%%%%%%%%%%%%%%%%%%%%%%%%%%%%%%%%%%%%%%%%%%%%%%%%%%%%%%%%%%%%%%%%%%%%%%%%%%%%%%

\documentclass{ifto-tex}

%%%%%%%%%%%%%%%%%%%%%%%%%%%%%%%%%%%%%%%%%%%%%%%%%%%%%%%%%%%%%%%%%%%%%%%%%%%%%%%%
% Inclusão de pacotes
%%%%%%%%%%%%%%%%%%%%%%%%%%%%%%%%%%%%%%%%%%%%%%%%%%%%%%%%%%%%%%%%%%%%%%%%%%%%%%%%

\usepackage{lmodern}			% Usa a fonte Latin Modern
\usepackage[T1]{fontenc}		% Selecao de codigos de fonte.
\usepackage[utf8]{inputenc}		% Codificacao do documento (conversão automática dos acentos)
\usepackage{indentfirst}		% Indenta o primeiro parágrafo de cada seção.
\usepackage{color}				% Controle das cores
\usepackage{graphicx}			% Inclusão de gráficos
\usepackage{microtype} 			% para melhorias de justificação
\usepackage{float}

% Uso da fonte Arial (IFTO)
\usepackage{helvet}
\renewcommand{\familydefault}{\sfdefault}

% Incluir pacotes adicionais, caso sejam necessários ....

%%%%%%%%%%%%%%%%%%%%%%%%%%%%%%%%%%%%%%%%%%%%%%%%%%%%%%%%%%%%%%%%%%%%%%%%%%%%%%%%
% Configuração das citações
%%%%%%%%%%%%%%%%%%%%%%%%%%%%%%%%%%%%%%%%%%%%%%%%%%%%%%%%%%%%%%%%%%%%%%%%%%%%%%%%
\usepackage[brazilian,hyperpageref]{backref}	 % Paginas com as citações na bibl
\usepackage[alf,
versalete,
abnt-emphasize = bf, % destaca o titulo em negrito;
abnt-etal-list = 3, % trabalhos com mais de 3 autores recebem et al.,;
abnt-etal-text = it, % escreve o et al., em italico;
abnt-and-type = &, % usa o carater '&' no lugar de 'e' para mais de um autor;
abnt-last-names = abnt, % trata sobrenomes 'estritamente' conforme a ABNT; e
abnt-repeated-author-omit = yes % autores com + de uma entrada recebem '____.'
]{abntex2cite}

% Configuração das referências bibliográficas
\renewcommand{\backref}{}
\renewcommand*{\backrefalt}[4]{	}

%%%%%%%%%%%%%%%%%%%%%%%%%%%%%%%%%%%%%%%%%%%%%%%%%%%%%%%%%%%%%%%%%%%%%%%%%%%%%%%%
% Informações da capa e da folha de rosto
%%%%%%%%%%%%%%%%%%%%%%%%%%%%%%%%%%%%%%%%%%%%%%%%%%%%%%%%%%%%%%%%%%%%%%%%%%%%%%%%

\titulo{TÍTULO DO TRABALHO: Subtítulo do trabalho}

\autor{Autor(a) do Trabalho}

\local{Paraíso do Tocantins, TO}

\data{2016}

% Alterar o nome do campus e do curso, caso houver necessidade
\instituicao{
	Instituto Federal de Educação, Ciência e Tecnologia do Tocantins - IFTO\\
	Campus Paraíso do Tocantins\\
	Curso Superior de Tecnologia em Gestão da Tecnologia da Informação
}

\tipotrabalho{TCC}

% Informar {Titulação}{Nome}
\orientador{Prof(a). Dr(a).}{Nome do(a) Professor(a) Orientador(a)}

% Alterar o preâmbulo conforme necessário
\preambulo{Trabalho de Conclusão de Curso
	apresentado como requisito parcial para
	obtenção do Título de tecnólogo do Curso Superior
	de Tecnologia em Gestão da Tecnologia da Informação do Instituto Federal do Tocantins,
	Campus Paraíso do Tocantins
}
% ---

% ---
% Configurações de aparência do PDF final


%%%%%%%%%%%%%%%%%%%%%%%%%%%%%%%%%%%%%%%%%%%%%%%%%%%%%%%%%%%%%%%%%%%%%%%%%%%%%%%%
% Outras configurações
%%%%%%%%%%%%%%%%%%%%%%%%%%%%%%%%%%%%%%%%%%%%%%%%%%%%%%%%%%%%%%%%%%%%%%%%%%%%%%%%

% Configuração da geração do PDF
\makeatletter
\hypersetup{
     	%pagebackref=true,
		pdftitle={\@title}, 
		pdfauthor={\@author},
    	pdfsubject={\imprimirpreambulo},
	    pdfcreator={LaTeX with abnTeX2},
		pdfkeywords={abnt}{latex}{abntex}{abntex2}{projeto de pesquisa}, 
		colorlinks=false,
		bookmarksdepth=4,
		pdfborder={0 0 0},
}
\makeatother


% O tamanho do parágrafo é dado por:
\setlength{\parindent}{1.3cm}

% Controle do espaçamento entre um parágrafo e outro:
\setlength{\parskip}{0.2cm}  % tente também \onelineskip

% compila o indice
\makeindex


%%%%%%%%%%%%%%%%%%%%%%%%%%%%%%%%%%%%%%%%%%%%%%%%%%%%%%%%%%%%%%%%%%%%%%%%%%%%%%%%
% CORPO DO TRABALHO ...
%%%%%%%%%%%%%%%%%%%%%%%%%%%%%%%%%%%%%%%%%%%%%%%%%%%%%%%%%%%%%%%%%%%%%%%%%%%%%%%%
\begin{document}

\selectlanguage{brazil}

% Retira espaço extra obsoleto entre as frases.
\frenchspacing 

% Inicializa a parte pre-textual
\pretextual

% Imprime a capa
\imprimircapa

% Imprime a folha de rosto
\imprimirfolhaderosto


% ------------------------------------------------------------------------------
% FOLHA DE APROVACAO
% ------------------------------------------------------------------------------
% Atenção: Alterar apenas a data e o nome dos convidados
\begin{folhadeaprovacao}
	
	\begin{center}
		{\ABNTEXchapterfont\bfseries\normalsize\imprimirautor}
		
		\vspace*{\fill}\vspace*{\fill}
		\begin{center}
			\ABNTEXchapterfont\bfseries\normalsize\imprimirtitulo
		\end{center}
		\vspace*{\fill}
		
		\hspace{.45\textwidth}
		\begin{minipage}{.5\textwidth}
			\imprimirpreambulo
		\end{minipage}%
		\vspace*{\fill}
	\end{center}
	
	Trabalho aprovado. \imprimirlocal, 24 de novembro de 2016:
	
	\assinatura{\textbf{\imprimirorientador} \\ Orientador} 
	\assinatura{\textbf{Prof. Dr. Fulano de tal} \\ Convidado 1}
	\assinatura{\textbf{Prof. Dr. Nome do outro convidado} \\ Convidado 2}
	
	\begin{center}
		\vspace*{0.5cm}
		{\normalsize\bfseries\imprimirlocal}
		\par
		{\normalsize\bfseries\imprimirdata}
		\vspace*{1cm}
	\end{center}
	
\end{folhadeaprovacao}


% ------------------------------------------------------------------------------
% AGRADECIMENTOS
% ------------------------------------------------------------------------------
\begin{agradecimentos}
	Folha em que o autor faz agradecimentos àqueles que contribuíram de maneira relevante para a realização do TCC...
\end{agradecimentos}


% ------------------------------------------------------------------------------
% EPIGRAFE
% ------------------------------------------------------------------------------
\begin{epigrafe}
	\vspace*{\fill}
	\begin{flushright}
		\textit{
			``Folha em que o autor apresenta uma citação,\\
			seguida da indicação da autoria, relacionada com a\\
			matéria tratada no corpo do trabalho.``\\
			(Nome do autor)
		}
	\end{flushright}
\end{epigrafe}


% ------------------------------------------------------------------------------
% RESUMO e ABSTRACT
% ------------------------------------------------------------------------------
\setlength{\absparsep}{18pt} % ajusta o espaçamento dos parágrafos do resumo

% Resumo em português
\begin{resumo}
	Apresentação concisa dos pontos relevantes do TCC, fornecendo uma visão rápida e clara dos objetivos, dos procedimentos metodológicos e das conclusões do trabalho.
	
	\textbf{Palavras-chave}: pc1, pc2, pc3 ...
\end{resumo}

% resumo em inglês (Abstract)
\begin{resumo}[Abstract]
	\begin{otherlanguage*}{english}
		English abstract version.
		
		\noindent 
		\textbf{Keywords}: kw1. kw2. kw ...
	\end{otherlanguage*}
\end{resumo}


% ------------------------------------------------------------------------------
% LISTA DE FIGURAS (Não altere nada aqui)
% ------------------------------------------------------------------------------
\pdfbookmark[0]{\listfigurename}{lof}
\listoffigures*
\cleardoublepage


% ------------------------------------------------------------------------------
% LISTA DE TABELAS (Não altere nada aqui)
% ------------------------------------------------------------------------------
\pdfbookmark[0]{\contentsname}{lot}
\listoftables*
\cleardoublepage

% ------------------------------------------------------------------------------
% LISTA DE SIGLAS E ABREVIATURAS
% ------------------------------------------------------------------------------
% Edite a lista de siglas conforme o modelo abaixo
\begin{siglas}
	\item[IBGE]{Instituto Brasileiro de Geografia e Estatística}
	\item[IFTO]{Instituto Federal do Tocantins}
	% Incluir as siglas aqui ...
\end{siglas}


% ------------------------------------------------------------------------------
% SUMÁRIO (Não altere nada aqui)
% ------------------------------------------------------------------------------
\pdfbookmark[0]{\contentsname}{toc}
\tableofcontents*
\cleardoublepage

% ------------------------------------------------------------------------------
% ELEMENTOS TEXTUAIS
% ------------------------------------------------------------------------------

% Introduz a parte textual
\textual

\chapter{Introdução}

	Parte inicial do texto que visa inserir o leitor no assunto trabalhado na sequência. Deve apresentar de forma geral o trabalho e, em especial, deve abordar o problema, os objetivos e a justificativa do trabalho.

	\section{Problema de pesquisa}
	
		Apresentação do problema que norteará a pesquisa para o TCC. A enunciação do problema deve ser,	preferencialmente, em forma de uma pergunta.
	
	\section{Justificativa}
	
		Texto construído com a intenção de mostrar	a relevância, a importância, a pertinência e a viabilidade do trabalho.
	
	\section{Objetivos}
	
		\subsection{Objetivo geral}
		
			O Objetivo Geral deve ter relação íntima com o problema de	pesquisa e deve apontar o rumo a ser percorrido para encontrar a resposta. Já os	Objetivos Específicos desdobram o Objetivo Geral nos passos necessários para executar o	Objetivo Geral. Os objetivos devem indicar	exatamente a ação a ser tomada.
		
		\subsection{Objetivos específicos}
		
			\begin{enumerate}
				\item Objetivo específico 1;
				\item Objetivo específico 2;
				\item Objetivo específico n.
			\end{enumerate}

\chapter{Revisão da Literatura}
	
	Trata-se de um texto que apresenta de forma	geral os fundamentos teóricos (ou bases	teóricas) e conceituais do trabalho. Deve destacar as principais obras e teorias da área em estudo. Deve estar bem referenciado.
	
	\begin{citacao}
		Exemplo de citação direta. \cite{Brandao2016}
	\end{citacao}
	
	Citação indireta \cite{Brandao2016}.
	
	Citando o autor \textit{inline}: \citeonline{Brandao2016}.
	
	Citando apenas o nome do autor: \citeauthoronline{Brandao2016}.
	
\chapter{Metodologia}
	
	Deve dizer como o trabalho será realizado. Aborda quatro componentes: descrição do foco do estudo; caracterização da pesquisa (tipo de pesquisa); plano de coleta de dados (técnicas e instrumentos de coleta de dados e informações afins); plano de análise dos dados (técnicas de sistematização e análise dos dados e as formas de apresentação dos resultados).
	
\chapter{Desenvolvimento...}
	
	Demais capítulos do desenvolvimento do trabalho...

\chapter{Considerações finais}

	Parte final do texto na qual são apresentadas as considerações finais acerca dos objetivos, das variáveis e do problema da pesquisa. Não usar esta seção para sumarizar os resultados (o que já foi feito no Resumo), mas destacar o progresso e as aplicações que o trabalho propicia. Enfatizar as limitações que persistem, apresentando, sempre que apropriado, sugestões para trabalhos futuros.

% Finaliza o bookmarking do PDF
\phantompart

% ------------------------------------------------------------------------------
% ELEMENTOS PÓS-TEXTUAIS
% ------------------------------------------------------------------------------

% Introduz a parte pós-textual
\postextual

% Insere as referências bibliográficas
\bibliography{bibliografia}

% ----------------- APÊNDICES -----------------
\begin{apendicesenv}
	
	\partapendices
	
	\chapter{Título do Apêndice}
	
		Documento ou texto “elaborado” pelo autor, inserido ou referido no trabalho a fim de complementar a sua argumentação. Para	não haver prejuízo à unidade nuclear do	trabalho, cópia integral do documento é	acrescida ao final.
		
	\chapter{Outro Apêndice}
	
		Outro documento ou texto “elaborado” pelo autor, inserido ou referido no trabalho a fim de complementar a sua argumentação. Para	não haver prejuízo à unidade nuclear do	trabalho, cópia integral do documento é	acrescida ao final.
	
	% Inserir outros apêndices caso seja necessário ...
	
\end{apendicesenv}


% ----------------- ANEXOS -----------------
\begin{anexosenv}
	
	\partanexos
	
	\chapter{Título do Anexo}
	
		Documento ou texto “não elaborado”	pelo autor, inserido ou referido no trabalho, a fim de fundamentar, comprovar ou ilustrar a sua argumentação. Igualmente, para não haver prejuízo à unidade nuclear do trabalho, cópia integral do documento é acrescida ao final do trabalho.
		
	\chapter{Outro Anexo}
	
		Outro documento ou texto “não elaborado”	pelo autor, inserido ou referido no trabalho, a fim de fundamentar, comprovar ou ilustrar a sua argumentação. Igualmente, para não haver prejuízo à unidade nuclear do trabalho, cópia integral do documento é acrescida ao final do trabalho.
	
\end{anexosenv}

\end{document}
